\documentclass[a4paper, twoside]{report}

%% Language and font encodings
\usepackage[english]{babel}
\usepackage[utf8x]{inputenc}
\usepackage[T1]{fontenc}
\usepackage{mathrsfs}

%% Sets page size and margins
\usepackage[a4paper,top=3cm,bottom=2cm,left=3cm,right=3cm,marginparwidth=1.75cm]{geometry}

\usepackage{amsmath}
\usepackage{amsthm}
\usepackage{graphicx}
\usepackage[]{listings}
\usepackage[colorinlistoftodos]{todonotes}
\usepackage[colorlinks=true, allcolors=blue]{hyperref}
\usepackage{comment}
\usepackage[numbers]{natbib}
\usepackage{fancyref}

\usepackage{report/title/titlestructure}

%% Utils
\usepackage{report/background/utils}

\title{A Deeper Dive into Relational Algebra by Way of Adjunctions}
\author{Matteo Bongiovanni}
\date{January 2023}
\supervisor{Dr.\ Nicolas Wu}
\secondmarker{Dr.\ Steffen van Bakel}
\documenttype{MEng Interim Report}
\university{Imperial College London}
\department{Department of Computing}
\degree{Joint Mathematics and Computing}

\begin{document}
\begin{comment}
This should include the project title and the name of the author of the report. You can also list the name of your supervisor if you wish. 
\end{comment}

\begin{titlepage}

\newcommand{\HRule}{\rule{\linewidth}{0.5mm}} % Defines a new command for the horizontal lines, change thickness here

%----------------------------------------------------------------------------------------
%	LOGO SECTION
%----------------------------------------------------------------------------------------

\includegraphics[width=8cm, draft=false]{title/logo.eps}\\[1cm] % Include a department/university logo - this will require the graphicx package
 
%----------------------------------------------------------------------------------------

\center\ % Center everything on the page

%----------------------------------------------------------------------------------------
%	HEADING SECTIONS
%----------------------------------------------------------------------------------------

\makeatletter
\textsc{\LARGE \@documenttype}\\[1.5cm]
\textsc{\Large \@university}\\[0.5cm]
\ifdefined\@degree\
\textsc{\large \@degree}\\[0.5cm]
\fi
\textsc{\large \@department}\\[0.5cm]
\makeatother

%----------------------------------------------------------------------------------------
%	TITLE SECTION
%----------------------------------------------------------------------------------------
\makeatletter
\HRule\\[0.4cm]
{ \huge \bfseries \@title}\\[0.4cm] % Title of your document
\HRule\\[1.5cm]
 
%----------------------------------------------------------------------------------------
%	AUTHOR SECTION
%----------------------------------------------------------------------------------------

\begin{minipage}{0.4\textwidth}
\begin{flushleft} \large
\emph{Author:}\\
\@author\ % Your name
\end{flushleft}
\end{minipage}
~\begin{minipage}{0.4\textwidth}
\begin{flushright} \large
\emph{Supervisor:} \\
\@supervisor\\[1.2em]

\ifdefined\@secondmarker\
\emph{Second Marker:} \\
\@secondmarker\
\fi

\end{flushright}
\end{minipage}\\[2cm]

%----------------------------------------------------------------------------------------
%	DATE SECTION
%----------------------------------------------------------------------------------------

{\large \@date}\\[2cm] % Date, change the \today to a set date if you want to be precise

\vfill % Fill the rest of the page with whitespace
\makeatother

\end{titlepage}

\begin{abstract}
In their distinguished paper ``Relational Algebra by Way of Adjunctions''\cite{RelationalAlgebraByWayOfAdjunctions} it was noted that the monadic structure of bulk types can help explain most of relational algebra. It was found that some operations, such as selections and projections, are more easily reasoned about; this paper completes the rigorous mathematical description of others such as relational join or grouping. The paper takes the novel stance of having a more broad view of adjunctions when dealing with the bulk types in order to make this extension. A theoretically more efficient implementation of joins is an immediate consequence of this approach and the first goal of the project is to benchmark the differences. The project can also be taken into the direction of using this reasoning to mathematically explain other known database query optimisations and potentially invent others.
\end{abstract}

\begin{comment}
The aim of the project interim report is multi-fold:

  1. To provide a document that your second marker can use as a basis for discussion on your project plan and progress to date.
  2. To show that you have considered the ethical implications of your project.
  3. To provide a substantial body of text, primarily the project background and related work, that you can use in your final report. 

By the time the interim report is due you should have a clearly defined project, understand well the motivation and issues to be addressed, know the background work in detail, have the main ideas for how to tackle the problem and have started the development. You should also have a plan for the remainder of the project and, importantly, how to evaluate the project.

The interim report should contain the following sections. An approximate page count is suggested for each section, but there are no hard limits either way:


You are free to write up any additional material that will appear in the final report, for example a section or chapter describing a significant component of the design/implementation that you have already completed.  Avoid any additional material that is not re-usable in the final report.

As always, use diagrams and examples (e.g. code) wherever appropriate.

If you need inspiration, take a look at the Distinguished Projects from previous years, focusing in particular at this stage on the introduction, background and evaluation sections.
\end{comment}

\tableofcontents
\listoffigures
\listoftables

\chapter{Introduction} % 1-3 pages
\begin{comment}
It’s a good idea to *try* to write the introduction to your final report early on in the project. However, you will find it hard, as you won’t yet have a complete story and you won’t know what your main contributions are going to be. However, the exercise is useful as it will tell you what you *don’t* yet know and thus what questions your project should aim to answer. For the interim report this section should be a short, succinct, summary of the project’s main objectives. Some of this material may be re-usable in your final report, but the chances are that your final introduction will be quite different.  You are therefore advised to keep this part of the interim report short, focusing on the following questions: What is the problem, why is it interesting and what’s your main idea for solving it?  (DON'T use those three questions as subheadings however!  The answers should emerge from what you write.)
\end{comment}

Databases are absolutely vital to modern day society and contain domain specific knowledge about anything from specialised images of eyes \todo{add citation here} to the structure of crystals \cite{CambridgeStructuralDatabase}. Since its conception many different data models describing how to hold the data in databases have emerged, including the relational model \cite{RelationalModel} and the semi-structured model\cite{DatabaseSystems}.

In this project we concern ourselves with the relational model (to be introduced in \fref{sec:relationalmodel}) as a way of modelling the database. This rich model has many methods of expressing queries, especially relational algebra and relational calculus, both with their strengths and weaknesses \cite{RelationalCalculus,RelationalModel}. \todo{they are equivalent though?} However, the favoured specification of the authors of \cite{RelationalAlgebraByWayOfAdjunctions} seemed to be through list comprehensions, a beautiful feature that ``provide for a concise and expressive notation for writing list-processing code''. \cite{MonadComprehensions} They eventually propose using GHC's extended list comprehension syntax specifically designed to help bridge the already close relationship between relational calculus and list comprehensions \cite{GHCListComprehension,ComprehensiveComprehensions} in order to avoid the significant theoretical performance hit.

It is widely noted that \emph{joins}, an integral operation in relational algebra are associated with inefficient implementations \cite{JoinProcessing}. It is easy to see why when considering the most general joins. However in their paper \cite{RelationalAlgebraByWayOfAdjunctions} they concern themselves with a specialised join called an \emph{equijoin}. As described in \fref{sec:joins} an equijoin is a specialised \emph{theta-join} -- a way of combining two relations based off of an arbitrary condition depending on the attributes of both relations. An integral part to the calculation of a theta-join is calculating the Cartesian product (all possible combinations of tuples of both relationships, as described in \fref{sec:products}). The algorithm must then filter every single tuple individually to check for equality of attributes! It is clear that this is wasteful for such an specialised join. As a more practical example consider the SQL program:
\begin{lstlisting}
  SELECT *
  FROM R, S
  WHERE R.a = S.b
\end{lstlisting}
This could be naively converted into a list comprehension with the following:
\[
  \left[\,(r, s)\;|\;r \leftarrow R,\;s \leftarrow S,\;r.a = s.b\,\right]
\]
where $(r, s)$ is seen as a single tuple whose attributes \attribute{\relation{R}.a} and \attribute{\relation{R}.b} are merged. \todo{fix example so that you do not need to see things}

 With this naive list comprehension implementation we effectively convert \equijoin{R}{\attribute{a}}{S}{\attribute{b}} to \select{\attribute{a} = \attribute{b}}{\relation{R} \times \relation{S}}, generating a relation with $|R||S|$ tuples in the process then filtering through each.

 This can much be much more efficiently implemented by viewing databases as indexed tables. We can index each relation by its associated attribute in the equijoin and merge the results -- localising the data required to in a cartesian product. This approach admits a linear time equijoin, if careful about comparison and projection functions. \cite{RelationalAlgebraByWayOfAdjunctions} With some mathematical tools explained in \fref{sec:gradedmonads} we can describe give these operations a monadic structure and therefore a comprehension syntax using the extended syntax discussed above.

 What this project adds to this story is a concrete demonstration of the improvement this solution offers.
 We will use the \emph{Haskell} and the list comprehensions and functions described above to implement a simple database querying software taking into account these key changes. Along with real world pragmatic data sources, benchmarking techniques found in \fref{sec:benchmarking} will be used to accurately measure and compare the efficiency difference between the two approaches -- with the new equijoin and without. This evidence would be very important to justify the use of these methods and the claims made in the paper \cite{RelationalAlgebraByWayOfAdjunctions}.
 With a concrete implementation, it could also provide insights into the downfall of remaining operations as well as those mentioned in the paper, potentially using profiling techniques to further analyse the performance bottlenecks, in order to inspire a theoretical approach at determining the issue as shown in \cite{RelationalAlgebraByWayOfAdjunctions} instead of an efficiency-driven optimisation approach.

In this chapter we will first introduce the mathematical concepts used in the construction of relational algebra in \cite{RelationalAlgebraByWayOfAdjunctions} as discrete topics in order to introduce some intuition and context. We will then use all that we covered to explore the evolution of the mathematical descriptions of database systems. Finally, an overview of other academic research used in this project's contributed will be presented.

\section{Category theory}
Category theory will be our main tool in describing the mathematical structure of different elements of our database systems, and relations between them. More generally, category theory can be seen as a way of taking the abstractions that algebra was built on to a higher level, an ``abstraction of abstractions''. You can find that just as easily as categories can help us in our domain, categories have a rich language and can describe many structures in mathematics ranging from groups and rings to matrices. 
\subsection{Categories}
\theoremstyle{definition}\newtheorem*{categorydef}{Category}
We will first take the purely mathematical introduction to category theory. We see that most structures in mathematics have similar key features: a collection of elements (typically with some rules governing them depending on the definition of the structure) and morphisms or transformations between them preserving the structure of elements. Notice, groups and group homomorphisms, rings and ring homomorphisms, topological spaces and continuous maps. This will be our inspiration while defining categories, ultimately the abstraction of these structures. \todo{introduce the non maths way of looking at it as well from the categories, types and structures book}
\begin{categorydef}
  A \emph{category} \cat{C} is a set\footnote{In more rigorous definitions one must be careful of defining the collections of objects as a set lest Russell's paradox comes into play}\todo{Make sure that this is correct.} of \emph{objects} \objs{C}, such as \obj{a}, \obj{b}, \obj{c}, and \emph{morphisms} (or \emph{arrows}) \morphs{C} between them, such as \morph{f}, \morph{g}. We require that:
  \begin{itemize}
    \item There are two operations; \emph{domain} which associates with every arrow \morph{f} an object $\obj{a} = \domain{f}$ and \emph{codomain} which associates with every arrow \morph{f} an object $\obj{b} = \codomain{f}$. We can now express this information as $\morph{f}: \obj{a} \to \obj{b}$.\footnote{Though we emphasise the distinction between a function and a morphism.}
    \item There is a composition rule between morphisms such that given $\morph{f}: \obj{a} \to \obj{b}$ and $\morph{g}: \obj{b} \to \obj{c}$, there is another arrow $\morph{g} \circ \morph{f}: \obj{a} \to \obj{c}$ in \morphs{C}.
    \item Composition of arrows is associative. That is, for an additional object \obj{d} and arrow $\morph{h}: \obj{c} \to \obj{d}$ the resulting morphisms $\morph{h} \circ \left(\morph{g} \circ \morph{f}\right)$ and $\left(\morph{h} \circ \morph{g}\right) \circ \morph{f}$ coincide in \morphs{C}.
    \item Every object \obj{a} is assigned an arrow $\id{a}: \obj{a} \to \obj{a}$ in \morphs{C}, called the \emph{identity morphism}.
    \item Composition with the identity morphism is the identity on morphisms. Explicitly, given the arrow $\morph{f}: \obj{a} \to \obj{b}$, we have $\morph{f} \circ \id{a} = \id{b} \circ \morph{f} = \morph{f}$.
  \end{itemize}
\end{categorydef}
\todo{introduce hom-set}

\subsection{Functors}
\subsection{Natural transformations}
\subsection{Adjunctions}
\todo{Describe why adjunctions are such a key character in this paper}
\todo{Add more information between here}
\subsection{Graded Monads}
\section{The relational model of a database}
We briefly describe the relational model of a database so that we can introduce the key operators we are modelling using category theory. \todo{not modelling, think of better word}
\subsection{Introduction to the relational model}
There are several different data models to choose from when designing a database that specify important aspects of the design such as the structure, operations and constraints on the data \cite{DatabaseSystems}. For this project we concern ourselves with the relational model and its associated algebra.

The relational model represents data through two dimensional tables, called \emph{relations}. Each relation contains several \emph{attributes}, denoting the columns of the table. We call the name of the relation and the set of its attributes a \emph{schema} and they are denoted by the name of the relation followed by the set of attributes in parentheses.\cite{DatabaseSystems} For instance: 
\begin{center}
\begin{verbatim}
  R(a1, a2,..., an)
\end{verbatim}
\end{center}
is a relation \relation{R} with $n$ attributes \attribute{a1}, \attribute{a2}, \ldots, \attribute{an}. A more practical example is the schema
\begin{figure}[!h]
\begin{verbatim}
  People(firstName, surname, age)
\end{verbatim}
\caption[Schema for the \relation{People} relation]{Example of a schema for the relation \relation{People}}
\label{fig:peopleSchema}
\end{figure}
This describes a schema for the relation \relation{People} and the attributes \attribute{firstName}, \attribute{surname} and \attribute{age}.
An empty table showing the relation can be found on \fref{tab:peopleRelationHeadings}.
\begin{table}[h]
  \centering
  \begin{tabular}{l|l|l}
    \attribute{firstName} & \attribute{surname} & \attribute{age} \\
    \hline\hline
    & &\\
  \end{tabular}
  \caption[\relation{People} relation's headings]{A tabular representation of the \relation{People} relation's attributes.}
  \label{tab:peopleRelationHeadings}
\end{table}

A row of the table\footnote{Excluding the row with the headings.} is called a \emph{tuple} and it consists of \emph{components}, one per attribute. A tuple is simply denoted by a comma separated list of its components within parentheses. For instance a tuple for the \relation{People} relation would be:
\begin{figure}[!h]
  \centering
  (\verb|Jim|, \verb|Smith|, 32).
\end{figure}

With this table we can now represent the \relation{People} in a tabular form in \fref{tab:peopleRelationWithTuple}:
\begin{table}[h]
  \centering
  \begin{tabular}{l|l|l}
    \attribute{firstName} & \attribute{surname} & \attribute{age} \\
    \hline\hline
    Jim & Smith & 32\\
  \end{tabular}
  \caption[\relation{People} relation with tuple]{A tabular representation of the \relation{People} relation and its associated tuple.}
  \label{tab:peopleRelationWithTuple}
\end{table}

In the relational model the constraint that each each component of a tuple must be an of an atomic type; meaning the component should reasonably not be a compound data type such as records, lists or sets. Furthermore, each attribute has its own associated \emph{domain}, specifying an elementary type that all components belonging to that column must take. The domain can be specified in the the schema using a colon as follows:
\begin{figure}[!h]
\end{figure}

It is natural now to see a relation as a set of tuples, and a database as one or more relations. The \emph{(relational) database schema} is the set of schemas the relations in the database adhere to.\cite{DatabaseSystems}
\todo{Do I need to talk about tuples being built up and a final mention of attributes}

Finally, we call a set of tuples for a relation an \emph{instance} of that relation as by the nature of database systems, it will eventually change. The current set of tuples is called the \emph{current instance}.\cite{DatabaseSystems}

\paragraph{A note on ordering} The distinction between lists and sets here is very important and has consequences. Firstly, note that a schema consists of a \textbf{set} of attributes, though in this case we do assign a ``standard'' ordering to them, typically following the ordering in the schema. Also note that when giving a tuple of a relation we do not also give the headings and thus some indication to which schema it belongs is necessary. Secondly, we introduced a relation as a \textbf{set} of tuples, in this case there is no standard ordering and thus invites many equivalent representations of the same relation.\cite{DatabaseSystems} In generality, the order of columns and rows do not matter, so long as they are consistent.

\subsection{Relational Algebra}
Equipped with the information above we can now introduce the domain of the project, relational algebra. Relational algebra views queries as the creation of a relation by operating on other relations.\cite{RelationalCalculus}

The sections below describe some of the more interesting, specialised operations on this algebra but it is definitely worth noting that as relations are sets \todo{bags?} of tuples, given matching schemas between relations, set operations are also valid operations and produce relations.\cite{RelationalModel} More specifically, you can find the union, intersection and differences of two relations just as you would with sets. \cite{DatabaseSystems}

For ease of demonstration we use a more populated instance of the \relation{People} relation as in \fref{tab:peopleRelationPopulated}.

\begin{table}[h]
  \centering
  \begin{tabular}{l|l|l}
    \attribute{firstName} & \attribute{surname} & \attribute{age} \\
    \hline\hline
    Jim & Smith & 32\\
    Herbert & Green & 34\\
    Emma & Smith & 23\\
  \end{tabular}
  \caption{Populated instance of the \relation{People} relation.}
  \label{tab:peopleRelationPopulated}
\end{table}
\subsubsection{Projections}
Projections select certain attributes of a relation, removing the others and collapsing duplicates \todo{I thought we needed multiplicity, check this}.\cite{RelationalModel} This can be visualised as only keeping certain columns in the table.

Given a list of $k$ attributes $L = \attribute{i_1}, \attribute{i_2}, \ldots, \attribute{i_k}$, we can specify a projection on an $n$-ary (containing $n$ attributes) relation \relation{R} as \proj{L}{R}. The result is a $k$-ary relation whose attributes are specified by $L$ \cite{RelationalModel} and by convention the ``standard order'' of attributes is also determined by the order of attributed in $L$.\cite{DatabaseSystems}

We can see a demonstration of the projection $\proj{\attribute{firstName},\;\attribute{age}}{People}$ in \fref{tab:peopleRelationProjection}:
\begin{table}[h]
  \centering
  \begin{tabular}{l|l}
    \attribute{firstName} & \attribute{age} \\
    \hline\hline
    Jim & 32\\
    Herbert & 34\\
    Emma & 23\\
  \end{tabular}
  \caption[Projection example on \relation{People} relation.]{An example projection of the people relation where the surname attribute has been removed.}
  \label{tab:peopleRelationProjection}
\end{table}

\subsubsection{Selections}
\subsubsection{Products}
\subsubsection{Joins}
\subsubsection{Note on permutations}
Permutations is another specialist operation in relational algebra, though not important to the scope of the project. For completion, despite the fact that relations are domain--unordered, their internal representation in computers is not and so permutation may be done for performance benefits despite no logical difference storing a relation and its permutations.\todo{Make sure I worded the performance benefits thing correctly}\cite{RelationalModel} Furthermore, permutation can be used (and is usually implied) to ensure that tuples with identical schemas differing only in ordering can have the normal set operations applied to them. \cite{DatabaseSystems}
\section{Evolution of database representation}
\subsection{Bags}
\paragraph{Characteristics of a database}We expect our database approximation to not be ordered and admit multiplicities and a finite bag of values is one of the simplest constructions that does so. Like a finite set, a bag contains a collection of unordered values. However, unlike a set, bags can contain duplicate elements \cite{RelationalAlgebraByWayOfAdjunctions}.  This multiplicity is key for processing non-idempotent aggregations. For instance, if summing up the ages of a database of people, without admitting multiplicity we would only sum each unique age once.
\subparagraph{Generalisation}Furthermore, going forward we generalise to bags of any types instead of the classical ``bags of records''. This also allows us to deal with intermediate tables that contain non-record values.

In \fref{tab:BagRelAlgOps} we summarise the implementation of relational algebra operators with bags
as their bulk type \cite{RelationalAlgebraByWayOfAdjunctions}.
\begin{table}[h]
    \centering
    \begin{tabular}{r|l}
        table of $V$ values & \bag{V} \\
        empty table & \emptybag \\
        singleton table & \singletonbag \\
        union of tables & $\bagunion{}{}$ \\
        Cartesian product of tables & $\times$ \\
        neutral element & $\lbag () \rbag$ \\
        projection $\projsymb{f}$ & $\bag{f}$ \\
        selection $\selectsymb{p}$ & $filter\ p$ \\
        aggregation in monoid $\monoid{M}$ & $reduce\ \monoid{M}$\\
    \end{tabular}
    \caption{Relational algebra operators implemented for bags}
    \label{tab:BagRelAlgOps}
\end{table}

\subsection{Indexed tables}
We want to move towards an indexed representation of our table in order to
equijoin by indexing. In this section we introduce the mathematical concepts required to define such an implementation.
\theoremstyle{definition}\newtheorem*{psetdef}{Pointed set}
\theoremstyle{definition}\newtheorem*{ppfuncdef}{Point-preserving function}
\theoremstyle{definition}\newtheorem*{mapdef}{Map}
\theoremstyle{definition}\newtheorem*{finitemapdef}{Finite map}
\begin{psetdef}\label{def:pset}
  A pointed set $\pset{A}{a}$ is a set $A$ with a distinguished element $a \in A$.
\end{psetdef}
We commonly refer to the distinguished element of a set $A$ as $null_A$ or, when not ambiguous to do so, $null$.
\begin{ppfuncdef}\label{def:ppfunc}
  Given two pointed sets $\pset{A}{null_A}$ and $\pset{B}{null_B}$, a total function $f: A \rightarrow B$ is point-preserving if $f(null_A) = null_B$.
\end{ppfuncdef}
\todo{See if there is a way to have better spacing in brackets}

\todo{Add in more mathematical detail for point preserving functions}

We now have the mathematical tools required to define a map. In its finite form a map is widely known in computer science by many other names such as a dictionary, association lists or key-value maps.

Let $\keyset$ be a set and $\valset$ a pointed set. To those already familiar with maps, it may help to think of $\keyset$ as keys and $\valset$ as values.
\begin{mapdef}
  A map of type $\map{\keyset}{\valset}$ is a total function from K to V.
\end{mapdef}
\begin{finitemapdef}
  A finite map of type \finitemap{\keyset}{\valset} is a map where only a finite number of keys are mapped to $null_\valset$ (where $null_\valset$ is the distinguished element of \valset). 
\end{finitemapdef}
The advantage of using a finite map in a database is to allow aggregation.
\todo{Understand why only semi-monoidal}
\todo{Introduced indexed tables}
\paragraph{Useful functions}{} \todo{Explain all the functions needed, such as
merge\label{sec:finitemapfuncs}}



\section{Benchmarking databases}\label{sec:benchmarking}
\section{Expanding the use of adjunctions}
\chapter{Project Plan} % 1-2 pages
\begin{comment}
You should explain what needs to be done in order to complete the project and roughly what you expect the timetable to be. Don’t forget to include the project write-up (the final report), as this is a major part of the exercise. It’s important to identify key milestones and also fall-back positions, in case you run out of time.  You should also identify what extensions could be added if time permits.  The plan should be complete and should include those parts that you have already addressed (make it clear how far you have progressed at the time of writing).  This material will *not* appear in the final report.
\end{comment}

The project has a clear starting point and a wide range of potential branches to eventually explore. This allows the project to venture down many different specific roots, ranging in application focused to theoretically based.
We outline the stages the project can take below.
\section{Initial research and literature review}
\section{Implementation of the differing equijoin algorithms}
\subsection{Implementation}
\subsection{Evaluation and analysis}
\section{Theoretical analysis of the remaining relational algebra}
\section{Reporting the project}
% Report due Monday 19 June 2022
\section{Possible extensions}
\subsection{Progress in optimising other aspects of relational algebra}
\subsection{A full implementation}
\subsection{Applications of related fields to relational calculus}

\chapter{Evaluation plan} % 1-2 pages
\begin{comment}
Project evaluation is very important, so it's important to think now about how you plan to measure success. For example, what functionality do you need to demonstrate?  What experiments to you need to undertake and what outcome(s) would constitute success?  What benchmarks should you use? How has your project extended the state of the art?  How do you measure qualitative aspects, such as ease of use?  These are the sort of questions that your project evaluation should address; this section should outline your plan.
\end{comment}
Evaluation of this project is important as the analyses need to be founded in order to have any sort of real benefit.

We split the evaluation of the implementation into two halves, the evaluation of the correctness and the evaluation of the analysis.

\section{Correctness}\label{sec:correctnessevaluation}
In order to evaluate the correctness it is vital to keep good coding practice throughout the project. It is not possible to analyse the performance of something incorrectly implemented, to justify the benefits of a different system entirely.

In order to do this, unit tests should be written during production, to ensure no developer error or mistake occurs by accident.

In order to further reduce mistakes, both bugs and theoretical misunderstandings if there are any the results of the two equijoin operations should be compared to ensure that the results are the same and thus they are being judged on the query interpreted in the same way.

\section{Analysis}
Evaluating the analysis is more difficult. Of course, as per usual benchmarking standards, each test would be run multiple times, changing whether in succession or not (in order to let any potential cache optimisations and ideal memory conditions occur). However these repeat results are conducted, the deviation should be closely monitored to ensure they are consistent.

It is difficult to suggest comparing deviations of other database systems to the one implemented, even of atomic operations, to ensure they follow a similar distribution of relative performance as databases are so mainstream that query optimisation is commonplace and may affect results in a way that cannot be predicted.
\chapter{Ethical issues} % 1-2 pages
\begin{comment}
What are the wider ethical, legal, professional and societal issues surrounding your project and the accompanying research? You should use the ethics checklist as the basis for this discussion. 
\end{comment}
The project itself (in its early stages) does not have a novel ideal as a useable output and thus is very neutral in its ethical impact. Instead it has the ability to recommend or withold recommendation of an earlier approach to database optimisations via implementation and benchmarking results, and in its final stages may recommend further optimisations and so the more interesting ethical discussion is around the ethics of database optimisations.

Databases are a necessity for most modern applications. Furthermore, database optimisations are especially important for large, scaling products or services. This means that improving this area of computer science has a very undiscriminated effect on the type of software products that can make use of it. Of course, just as any private company could make use of the newer more efficient techniques, any military hosting a database could too. It is also worth noting the implementation is in a functional programming language, a rising paradigm \todo{cite} typically associated with reliability \todo{add another quality} \todo{Add citation for this} which aligns very well with the real-time and safety critical nature of military applications. \todo{cite that these are important for military applications}

In fact, it could be argued that any findings could benefit society at large in an environmental standpoint. The heart of the project is in database optimisations, which in turn reduce the amount of time required to process queries. \todo{research this} One could extrapolate, that without much of an increased processing complexity and resource consumption, this would lead to a reduction in the energy required for these database queries. Given how commonplace database queries are \todo{find stats} this could have a nontrivial impact.

It is worth noting that the early stages of the project involves implementing the ideas created in \cite{RelationalAlgebraByWayOfAdjunctions}. This causes no legal issues as there is no protection on an idea, lacking the tangibility (and ability to be patented) \todo{See exact laws}. On an ethical level, despite implementations, improvements and building on work is commonplace, encouraged and vital to the academic community, one of the main contributors of the paper is my supervisor and I interpret this as implicit consent. With good referencing practice, plagiarism would also be completely avoided.

There is an interesting discussion to be had on the ethical impact of the benchmarking dataset to be used in the early stages of the project. In order to evaluate the suggestions in \cite{RelationalAlgebraByWayOfAdjunctions} a database system will need to be implemented, but more consequentially one or more data sources will need to be used to evaluate the queries. I do not think that creating a novel data source will be very useful or worthwhile so I will need to acquire at least one somehow, raising at least two issues.
\subparagraph*{Data protection} The immediate issue that springs to mind would be whether the data source contains personal data and if so the subsequent data protection methods required to fulfil my legal and ethical obligation. To avoid this issue for primary data sources, I will not collect data when creating a benchmarking database, instead opting to randomly generate records. For data sources I acquire, they will all be in public domain already, and avoid sensitive personal data. This should minimise the risk to individuals affected. \todo{See if this is enough} \todo{Cite data protection laws}
\subparagraph*{Database rights} Another consideration when outsourcing database collections, is the extent to which those who have created the database are protected. The creation of a database, although not creative, has been recognised as taking significant work and so those that create the database have a form of copyright protection on their work. \todo{Make sure that this is accurate, cite if so} This risk will be mitigated by ensuring that all databases used are in the public domain and the correct licenses are acquired.

\todo{Potentially comment on the distribution of my implementation}

\bibliographystyle{unsrtnat}
% DoC uses the Vancouver Referencing Format.
\bibliography{report/bibs/interim}

\end{document}