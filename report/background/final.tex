\chapter{Background}
\begin{comment}
The background section of the report should set the project into context by relating it to existing published work which you read at the start of the project when your approach and methods were being considered. There are usually many ways of solving a given problem, and you shouldn't just pick one at random. Describe and evaluate as many alternative approaches as possible. The published work may be in the form of research papers found in the academic literature, articles, text books, technical manuals, or even existing software or hardware of which you have had hands-on experience. Your must acknowledge the sources of your inspiration. You are expected to have seen and thought about other people's ideas; your contribution will be putting them into practice in some other context. However, avoid plagiarism: if you take another person's work as your own and do not cite your sources of information/inspiration you are being dishonest. When referring to other pieces of work, cite the sources where they are referred to or used, rather than just listing them at the end. Accidental plagiarism or not knowing how to cite and reference is not a valid reason for plagiarism. Make sure you read and digest the Department's plagiarism document .

In writing the Background chapter you must demonstrate your ability to analyse, synthesise and apply critical judgement. Analysis is shown by explaining how the proposed solution operates in your own words as well as its benefits and consequences. Synthesis is shown through the organisation of your Related Work section and through identifying and generalising common aspects across different solutions. Critical judgement is shown by discussing the limitations of the solutions proposed both in terms of their disadvantages and limits of applicability.

Typically you can look for Background work using different search engines including:

Google Scholar
IEEExplore
ACM Digital Library
Citeseer
Science Direct
Note 1: Often the terms Background, Literature Review, Related Work and State of the Art are used interchangeably.
Note 2: Keyword search is wonderful, but you need the right Keywords.
Note 2: IEEExplore, ACM Digital Library and Science Direct may require you to be on the College network to download the PDF versions of papers. If at home, use VPN.
\end{comment}

\section{The relational model of a database}
We briefly describe the relational model of a database so that we can introduce the key operators we are modelling using category theory. \todo{not modelling, think of better word}
\subsection{Introduction to the relational model}
There are several different data models to choose from when designing a database that specify important aspects of the design such as the structure, operations and constraints on the data \cite{DatabaseSystems}. For this project we concern ourselves with the relational model and its associated algebra.

The relational model represents data through two dimensional tables, called \emph{relations}. Each relation contains several \emph{attributes}, denoting the columns of the table. We call the name of the relation and the set of its attributes a \emph{schema} and they are denoted by the name of the relation followed by the set of attributes in parentheses.\cite{DatabaseSystems} For instance: 
\begin{center}
\begin{verbatim}
  R(a1, a2,..., an)
\end{verbatim}
\end{center}
is a relation \relation{R} with $n$ attributes \attribute{a1}, \attribute{a2}, \ldots, \attribute{an}. A more practical example is the schema
\begin{figure}[!h]
\begin{verbatim}
  People(firstName, surname, age)
\end{verbatim}
\caption[Schema for the \relation{People} relation]{Example of a schema for the relation \relation{People}}
\label{fig:peopleSchema}
\end{figure}
This describes a schema for the relation \relation{People} and the attributes \attribute{firstName}, \attribute{surname} and \attribute{age}.
An empty table showing the relation can be found on \fref{tab:peopleRelationHeadings}.
\begin{table}[h]
  \centering
  \begin{tabular}{l|l|l}
    \attribute{firstName} & \attribute{surname} & \attribute{age} \\
    \hline\hline
    & &\\
  \end{tabular}
  \caption[\relation{People} relation's headings]{A tabular representation of the \relation{People} relation's attributes.}
  \label{tab:peopleRelationHeadings}
\end{table}

A row of the table\footnote{Excluding the row with the headings.} is called a \emph{tuple} and it consists of \emph{components}, one per attribute. A tuple is simply denoted by a comma separated list of its components within parentheses. For instance a tuple for the \relation{People} relation would be:
\begin{figure}[!h]
  \centering
  (\verb|Jim|, \verb|Smith|, 32).
\end{figure}

With this table we can now represent the \relation{People} in a tabular form in \fref{tab:peopleRelationWithTuple}:
\begin{table}[h]
  \centering
  \begin{tabular}{l|l|l}
    \attribute{firstName} & \attribute{surname} & \attribute{age} \\
    \hline\hline
    Jim & Smith & 32\\
  \end{tabular}
  \caption[\relation{People} relation with tuple]{A tabular representation of the \relation{People} relation and its associated tuple.}
  \label{tab:peopleRelationWithTuple}
\end{table}

In the relational model the constraint that each each component of a tuple must be an of an atomic type; meaning the component should reasonably not be a compound data type such as records, lists or sets. Furthermore, each attribute has its own associated \emph{domain}, specifying an elementary type that all components belonging to that column must take. The domain can be specified in the the schema using a colon as follows:
\begin{figure}[!h]
\end{figure}

It is natural now to see a relation as a set of tuples, and a database as one or more relations. The \emph{(relational) database schema} is the set of schemas the relations in the database adhere to.\cite{DatabaseSystems}
\todo{Do I need to talk about tuples being built up and a final mention of attributes}

Finally, we call a set of tuples for a relation an \emph{instance} of that relation as by the nature of database systems, it will eventually change. The current set of tuples is called the \emph{current instance}.\cite{DatabaseSystems}

\paragraph{A note on ordering} The distinction between lists and sets here is very important and has consequences. Firstly, note that a schema consists of a \textbf{set} of attributes, though in this case we do assign a ``standard'' ordering to them, typically following the ordering in the schema. Also note that when giving a tuple of a relation we do not also give the headings and thus some indication to which schema it belongs is necessary. Secondly, we introduced a relation as a \textbf{set} of tuples, in this case there is no standard ordering and thus invites many equivalent representations of the same relation.\cite{DatabaseSystems} In generality, the order of columns and rows do not matter, so long as they are consistent.

\subsection{Relational Algebra}
Equipped with the information above we can now introduce the domain of the project, relational algebra. Relational algebra views queries as the creation of a relation by operating on other relations.\cite{RelationalCalculus}

The sections below describe some of the more interesting, specialised operations on this algebra but it is definitely worth noting that as relations are sets \todo{bags?} of tuples, given matching schemas between relations, set operations are also valid operations and produce relations.\cite{RelationalModel} More specifically, you can find the union, intersection and differences of two relations just as you would with sets. \cite{DatabaseSystems}

For ease of demonstration we use a more populated instance of the \relation{People} relation as in \fref{tab:peopleRelationPopulated}.

\begin{table}[h]
  \centering
  \begin{tabular}{l|l|l}
    \attribute{firstName} & \attribute{surname} & \attribute{age} \\
    \hline\hline
    Jim & Smith & 32\\
    Herbert & Green & 34\\
    Emma & Smith & 23\\
  \end{tabular}
  \caption{Populated instance of the \relation{People} relation.}
  \label{tab:peopleRelationPopulated}
\end{table}
\subsubsection{Projections}
Projections select certain attributes of a relation, removing the others and collapsing duplicates \todo{I thought we needed multiplicity, check this}.\cite{RelationalModel} This can be visualised as only keeping certain columns in the table.

Given a list of $k$ attributes $L = \attribute{i_1}, \attribute{i_2}, \ldots, \attribute{i_k}$, we can specify a projection on an $n$-ary (containing $n$ attributes) relation \relation{R} as \proj{L}{R}. The result is a $k$-ary relation whose attributes are specified by $L$ \cite{RelationalModel} and by convention the ``standard order'' of attributes is also determined by the order of attributed in $L$.\cite{DatabaseSystems}

We can see a demonstration of the projection $\proj{\attribute{firstName},\;\attribute{age}}{People}$ in \fref{tab:peopleRelationProjection}:
\begin{table}[h]
  \centering
  \begin{tabular}{l|l}
    \attribute{firstName} & \attribute{age} \\
    \hline\hline
    Jim & 32\\
    Herbert & 34\\
    Emma & 23\\
  \end{tabular}
  \caption[Projection example on \relation{People} relation.]{An example projection of the people relation where the surname attribute has been removed.}
  \label{tab:peopleRelationProjection}
\end{table}

\subsubsection{Selections}
\subsubsection{Products}
\subsubsection{Joins}
\subsubsection{Note on permutations}
Permutations is another specialist operation in relational algebra, though not important to the scope of the project. For completion, despite the fact that relations are domain--unordered, their internal representation in computers is not and so permutation may be done for performance benefits despite no logical difference storing a relation and its permutations.\todo{Make sure I worded the performance benefits thing correctly}\cite{RelationalModel} Furthermore, permutation can be used (and is usually implied) to ensure that tuples with identical schemas differing only in ordering can have the normal set operations applied to them. \cite{DatabaseSystems}
\section{Benchmarking databases}
Databases are pervasive to modern society and thus standards have arisen over
the last few decades to ensure that customers are able to pick their preferred
DBMS vendor.

At the top level, database benchmarks are classified into three categories: industry-standard, vendor and
customer-application \cite{PractitionersIntroduction}. 
These classifications are usually motivated by intention of the benchmark
instead of structure of the database management system; there is no shortage of
papers emphasising the importance of domain-specific benchmarks for applications
\cite{PractitionersIntroduction, BenchmarkHandbook} and depending on the
risk/performance tolerance of the application it may be necessary to consider
results from all three categories.

\paragraph{Vendor benchmark} A vendor database benchmark is used by
the database vendor during the production of the database management system.
This benchmark usually serves multiple purposes not just limited to the design
of the system. Of course, it is often used to highlight any performance
bottlenecks driving the design of the system internally but it usually doubles
up and acts as the knowledge base for the marketing of the system. Vendor
benchmarks are usually characterised by a more comprehensive suite of
tests to generate the insights capable of driving the direction of the
product \cite{PractitionersIntroduction}.

\paragraph{Industry-standard benchmark} An industry-standard benchmark is a set
benchmarking suite designed independently to any vendor or solution. It is
designed to allow for a fair comparison between different vendors and has been
shown to increase competition between vendors \cite{Wisconsin2}. As much of
this review will show, many industry-standard benchmarks have been developed
to give results for a wide range of applications of databases for much more
relevant and specialised results; and, although many benchmarks have similar
metrics \cite{SetQueryBenchmark, DebitCredit}, just selecting what metrics to
show consumers is not a trivial task \cite{DebitCredit}. When Gray's paper
\cite{BenchmarkHandbook} was written, it was noted that these benchmarks were
becoming so popular that vendors were also beginning to report their results
with their marketing.

\paragraph{Consumer-application benchmark} This type of benchmark refers to any
benchmarking that a customer would run, typically to choose between different
vendors for their application. This kind of testing can be critical for a
performance-sensitive application \cite{PractitionersIntroduction} and is often
done to test the performance of the database under a specific installation
profile \cite{DoingYourOwnBenchmark} or loads. This is a very specialised
requirements based benchmarking.

The review now turns to different domain-specific benchmarking of databases in
order to comment on and analyse any common structures found when designing
databases. This will help inform major design decisions of this project when
deciding what aspects to include in the benchmarking of the alternative join
syntax.

Despite the innumerable use cases for databases in the modern world, many
applications require and prioritise similar values when choosing a solution.
Furthermore, as most industrial-standard database benchmarks have been designed in
order to allow comparison for specific domain requirements, you will commonly
find database benchmarks that are designed with one of two different
applications in mind: \emph{transaction processing} or \emph{decision support}
\cite{PractitionersIntroduction}; it is worth mentioning that other application
types exist, however, such as \emph{document search} and \emph{direct marketing}
\cite{SetQueryBenchmark}. The application types of transaction processing and
decision support are so pervasive that they are commonly also used to partition
other application types, for instance OLTP (Online transaction processing) and
OLAP (Online analytical processing) are specialised for online processing but
share many comparative similarities as transaction processing and decision
support respectively \cite{OLTP-Oracle}.

This review will provide a brief overview of both
the main database application types, then focus more strongly on decision
support benchmarks as ad-hoc query processing is more heavily tested in these
benchmarks and thus more relevant to the project.

\subsection{Transaction processing}
Transaction processing is characterised by a large number of
update-intensive requests
\cite{PractitionersIntroduction}. It is clear that this type of environment
demands an emphasis on throughput and integrity. For instance, a bank would
likely have an \emph{online transaction processing system} or OLTP system in
place and it clear that the ability to deal with a large number of transactions
in a short period of time and for the accuracy if the information given is
paramount. There are many well known benchmarks to test these types of systems,
including \emph{DebitCredit}, \emph{TPC-C} and \emph{TPC-E} \cite{TPC-OLTP}.
These benchmarks usually measure transactions per second.

\subsubsection{Transaction integrity}
We briefly note what we mean by the integrity of a transaction. A transaction
simply is an input to the database that must be considered as one unit of work
\cite{ComputerScienceDictionary} and completed independently to any concurrent
actions occurring. This behaviour can be described a set of
properties typically remembered as the acronym \emph{ACID}
\cite{ComputerScienceDictionary, PractitionersIntroduction}.
\paragraph{atomicity}
\paragraph{consistency}
\paragraph{isolation}
\paragraph{durability}

\subsection{Decision processing}
\subsubsection{Wisconsin benchmark}

\subsubsection{Single-User Decision Support} % SUDS

\subsubsection{The Set Query Benchmark}

\subsection{Best practices benchmarking}

\section{Category theory}
Category theory will be our main tool in describing the mathematical structure of different elements of our database systems, and relations between them. More generally, category theory can be seen as a way of taking the abstractions that algebra was built on to a higher level, an ``abstraction of abstractions''. You can find that just as easily as categories can help us in our domain, categories have a rich language and can describe many structures in mathematics ranging from groups and rings to matrices. 
\subsection{Categories}
\theoremstyle{definition}\newtheorem*{categorydef}{Category}
We will first take the purely mathematical introduction to category theory. We see that most structures in mathematics have similar key features: a collection of elements (typically with some rules governing them depending on the definition of the structure) and morphisms or transformations between them preserving the structure of elements. Notice, groups and group homomorphisms, rings and ring homomorphisms, topological spaces and continuous maps. This will be our inspiration while defining categories, ultimately the abstraction of these structures. \todo{introduce the non maths way of looking at it as well from the categories, types and structures book}
\begin{categorydef}
  A \emph{category} \cat{C} is a set\footnote{In more rigorous definitions one must be careful of defining the collections of objects as a set lest Russell's paradox comes into play}\todo{Make sure that this is correct.} of \emph{objects} \objs{C}, such as \obj{a}, \obj{b}, \obj{c}, and \emph{morphisms} (or \emph{arrows}) \morphs{C} between them, such as \morph{f}, \morph{g}. We require that:
  \begin{itemize}
    \item There are two operations; \emph{domain} which associates with every arrow \morph{f} an object $\obj{a} = \domain{f}$ and \emph{codomain} which associates with every arrow \morph{f} an object $\obj{b} = \codomain{f}$. We can now express this information as $\morph{f}: \obj{a} \to \obj{b}$.\footnote{Though we emphasise the distinction between a function and a morphism.}
    \item There is a composition rule between morphisms such that given $\morph{f}: \obj{a} \to \obj{b}$ and $\morph{g}: \obj{b} \to \obj{c}$, there is another arrow $\morph{g} \circ \morph{f}: \obj{a} \to \obj{c}$ in \morphs{C}.
    \item Composition of arrows is associative. That is, for an additional object \obj{d} and arrow $\morph{h}: \obj{c} \to \obj{d}$ the resulting morphisms $\morph{h} \circ \left(\morph{g} \circ \morph{f}\right)$ and $\left(\morph{h} \circ \morph{g}\right) \circ \morph{f}$ coincide in \morphs{C}.
    \item Every object \obj{a} is assigned an arrow $\id{a}: \obj{a} \to \obj{a}$ in \morphs{C}, called the \emph{identity morphism}.
    \item Composition with the identity morphism is the identity on morphisms. Explicitly, given the arrow $\morph{f}: \obj{a} \to \obj{b}$, we have $\morph{f} \circ \id{a} = \id{b} \circ \morph{f} = \morph{f}$.
  \end{itemize}
\end{categorydef}
\todo{introduce hom-set}

\subsection{Functors}
\subsection{Natural transformations}
\subsection{Adjunctions}
\todo{Describe why adjunctions are such a key character in this paper}
\todo{Add more information between here}
\subsection{Graded Monads}
\section{Evolution of database representation}
\subsection{Bags}
\paragraph{Characteristics of a database}We expect our database approximation to not be ordered and admit multiplicities and a finite bag of values is one of the simplest constructions that does so. Like a finite set, a bag contains a collection of unordered values. However, unlike a set, bags can contain duplicate elements \cite{RelationalAlgebraByWayOfAdjunctions}.  This multiplicity is key for processing non-idempotent aggregations. For instance, if summing up the ages of a database of people, without admitting multiplicity we would only sum each unique age once.
\subparagraph{Generalisation}Furthermore, going forward we generalise to bags of any types instead of the classical ``bags of records''. This also allows us to deal with intermediate tables that contain non-record values.

In \fref{tab:BagRelAlgOps} we summarise the implementation of relational algebra operators with bags
as their bulk type \cite{RelationalAlgebraByWayOfAdjunctions}.
\begin{table}[h]
    \centering
    \begin{tabular}{r|l}
        table of $V$ values & \bag{V} \\
        empty table & \emptybag \\
        singleton table & \singletonbag \\
        union of tables & $\bagunion{}{}$ \\
        Cartesian product of tables & $\times$ \\
        neutral element & $\lbag () \rbag$ \\
        projection $\projsymb{f}$ & $\bag{f}$ \\
        selection $\selectsymb{p}$ & $filter\ p$ \\
        aggregation in monoid $\monoid{M}$ & $reduce\ \monoid{M}$\\
    \end{tabular}
    \caption{Relational algebra operators implemented for bags}
    \label{tab:BagRelAlgOps}
\end{table}

\subsection{Indexed tables}
We want to move towards an indexed representation of our table in order to
equijoin by indexing. In this section we introduce the mathematical concepts required to define such an implementation.
\theoremstyle{definition}\newtheorem*{psetdef}{Pointed set}
\theoremstyle{definition}\newtheorem*{ppfuncdef}{Point-preserving function}
\theoremstyle{definition}\newtheorem*{mapdef}{Map}
\theoremstyle{definition}\newtheorem*{finitemapdef}{Finite map}
\begin{psetdef}\label{def:pset}
  A pointed set $\pset{A}{a}$ is a set $A$ with a distinguished element $a \in A$.
\end{psetdef}
We commonly refer to the distinguished element of a set $A$ as $null_A$ or, when not ambiguous to do so, $null$.
\begin{ppfuncdef}\label{def:ppfunc}
  Given two pointed sets $\pset{A}{null_A}$ and $\pset{B}{null_B}$, a total function $f: A \rightarrow B$ is point-preserving if $f(null_A) = null_B$.
\end{ppfuncdef}
\todo{See if there is a way to have better spacing in brackets}

\todo{Add in more mathematical detail for point preserving functions}

We now have the mathematical tools required to define a map. In its finite form a map is widely known in computer science by many other names such as a dictionary, association lists or key-value maps.

Let $\keyset$ be a set and $\valset$ a pointed set. To those already familiar with maps, it may help to think of $\keyset$ as keys and $\valset$ as values.
\begin{mapdef}
  A map of type $\map{\keyset}{\valset}$ is a total function from K to V.
\end{mapdef}
\begin{finitemapdef}
  A finite map of type \finitemap{\keyset}{\valset} is a map where only a finite number of keys are mapped to $null_\valset$ (where $null_\valset$ is the distinguished element of \valset). 
\end{finitemapdef}
The advantage of using a finite map in a database is to allow aggregation.
\todo{Understand why only semi-monoidal}
\todo{Introduced indexed tables}
\paragraph{Useful functions}{} \todo{Explain all the functions needed, such as
merge\label{sec:finitemapfuncs}}

 
\section{Implementation in Haskell}
\subsection{Category theoretical  concepts}
\subsection{Comprehension syntax}
\subsection{Benchmarking}
\subsubsection{Difficulties}
\paragraph{Lazy evaluation}
\paragraph{Garbage collection}

