\section{The relational model of a database}
We briefly describe the relational model of a database so that we can introduce the key operators we are modelling using category theory. \todo{not modelling, think of better word}
\subsection{Introduction to the relational model}
There are several different data models to choose from when designing a database that specify important aspects of the design such as the structure, operations and constraints on the data \cite{DatabaseSystems}. For this project we concern ourselves with the relational model and its associated algebra.

The relational model represents data through two dimensional tables, called \emph{relations}. Each relation contains several \emph{attributes}, denoting the columns of the table. We call the name of the relation and the set of its attributes a \emph{schema} and they are denoted by the name of the relation followed by the set of attributes in parentheses.\cite{DatabaseSystems} For instance: 
\begin{center}
\begin{verbatim}
  R(a1, a2,..., an)
\end{verbatim}
\end{center}
is a relation \verb|R| with $n$ attributes \verb|a1|, \verb|a2|, \ldots, \verb|an|. A more practical example is the schema
\begin{figure}[!h]
\begin{verbatim}
  People(firstName, surname, age)
\end{verbatim}
\caption[Schema for the People relation]{Example of a schema for the relation People}
\label{fig:peopleSchema}
\end{figure}
This describes a schema for the relation \verb|People| and the attributes \verb|firstName|, \verb|surname| and \verb|age|.
An empty table showing the relation can be found on \fref{tab:peopleRelationHeadings}.
\begin{table}[h]
  \centering
  \begin{tabular}{l|l|l}
    \verb|firstName| & \verb|surname| & \verb|age| \\
    \hline\hline
    & &\\
  \end{tabular}
  \caption{A tabular representation of the People relation.}
  \label{tab:peopleRelationHeadings}
\end{table}
\todo{Add verb to People in the caption}

\subsection{Relational Algebra}
\subsubsection{Projections}
\subsubsection{Selections}
\subsubsection{Products}
\subsubsection{Joins}