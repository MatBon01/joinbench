\section{Category theory}
Category theory will be our main tool in describing the mathematical structure of different elements of our database systems, and relations between them. More generally, category theory can be seen as a way of taking the abstractions that algebra was built on to a higher level, an ``abstraction of abstractions''. You can find that just as easily as categories can help us in our domain, categories have a rich language and can describe many structures in mathematics ranging from groups and rings to matrices. 
\subsection{Categories}
We will first take the purely mathematical introduction to category theory. We see that most structures in mathematics have similar key features: a collection of elements (typically with some rules governing them depending on the definition of the structure) and morphisms or transformations between them preserving the structure of elements. Notice, groups and group homomorphisms, rings and ring homomorphisms, topological spaces and continuous maps. This will be our inspiration while defining categories, ultimately the abstraction of these structures.

\subsection{Functors}
\subsection{Natural transformations}
\subsection{Adjunctions}
\todo{Describe why adjunctions are such a key character in this paper}
\todo{Add more information between here}
\subsection{Graded Monads}