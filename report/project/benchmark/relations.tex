\section{Relations}
Following are some characteristics of decision support systems \cite{IntroToDatabaseSystems}
that make their structure and benchmarks particularly relevant to this
project.
% Paper in this context referes to relational algebra by way of
% adjunctions

\subparagraph{Join complexity} Queries in decision support systems typically
have more complicated design - this is by virtue of the need to access many
kinds of facts. Although the paper \cite{RelationalAlgebraByWayOfAdjunctions}
does not recommend any advice for dealing with multiple joins on a practical
industrial scale (as \cite{IntroToDatabaseSystems} describes with industry
standard \emph{prejoin}s), its work in central on the efficiency of joins
(albeit under comprehension syntax).

\subparagraph{Ad hoc queries} Decision support frameworks usually rely on ad hoc
queries more than other applications and as such their benchmarks test its
ability to deal with these queries more than other applications
\cite{SetQueryBenchmark, PractitionersIntroduction}.

\subparagraph{Integrity unimportant} The paper does not provide any way of
updating the table, only query methods. This is similar to the aspect of
decision support systems who assume that the data is assumed to be correct and
do not deal with many updates, therefore little or no emphasis is placed on
testing the integrity of the system. This helps construct a benchmark that more
accurately reflects the discoveries of the paper.
