\section{Benchmark workflow}\label{sec:benchmark:workflow}
In order to efficiently benchmark and analyse the performance of the database
management system a workflow system was created to automate many of the tasks
involved. This section will briefly describe the workflow and its creation.

There are multiple steps involved in the benchmarking of the database management
systems. The first step is to create synthetic version of the
\relation{JOINBENCH} relation up to the specifications of the benchmark.
Alternatively, one relation at least as large as the highest cardinality
interested in benchmarking can be made and then selections the
\relationAttribute{unique} attributes can be used to alter the cardinality;
however, this involved extra computation on the side of the system to benchmark
and therefore I decided to create a synthetic relation for each tuple count.
After the relations are created, they must be parsed into the Haskell program
and loaded into the database. All the queries can then be run and benchmarked.
Finally, the data is processed and the relevant statistics and figures are
extracted and created.
