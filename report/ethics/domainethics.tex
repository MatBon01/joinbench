\section{Domain ethics}
To my assessment the conclusions of this paper will have a neutral impact on
society. The domain of the research, database management systems, is pervasive in nature and therefore not
applicable to any particular cause.

Databases are a necessity for most modern applications. Furthermore, database
optimisations are especially important for large, scaling products or services.
This means that improving this area of computer science has a very
indiscriminate effect on the type of software products that can make use of it.
Of course, just as any private company could make use of the newer more
efficient techniques, any military hosting a database could too. It is also
worth noting the implementation is in a functional programming language, a
rising paradigm \cite{LanguagePopularity} typically associated with reliability
\cite{RealWorldHaskell} which aligns very well with the real-time and safety critical nature of military applications.

In fact, it could be argued that any findings could benefit society at large in an environmental standpoint. The heart of the project is in database optimisations, which in turn reduce the amount of time required to process queries. One could extrapolate, that without much of an increased processing complexity and resource consumption, this would lead to a reduction in the energy required for these database queries. Given how commonplace database queries are this could have a non-trivial impact.
On the other hand, this research is on a niche subset of database
management systems, only focusing not only on database management systems in
Haskell, a language not very popular for such applications, but on queries
handwritten using list comprehension notation which suggest that any
findings would not be applicable to such scenarios anyway.
