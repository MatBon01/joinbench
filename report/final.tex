\documentclass[a4paper, twoside]{report}

%% Language and font encodings
\usepackage[english]{babel}
\usepackage[utf8x]{inputenc}
\usepackage[T1]{fontenc}
\usepackage{mathrsfs}

%% Sets page size and margins
\usepackage[a4paper,top=3cm,bottom=2cm,left=3cm,right=3cm,marginparwidth=1.75cm]{geometry}

\usepackage{amsthm}
\usepackage{graphicx}
\usepackage[]{listings}
\usepackage{booktabs}
\usepackage[colorinlistoftodos]{todonotes}
\usepackage[colorlinks=true, allcolors=blue]{hyperref}
\usepackage{comment}
\usepackage[numbers]{natbib}
\usepackage{fancyref}
\usepackage{wrapfig}

%% Math
\usepackage{mathtools}
\usepackage{stmaryrd}

\usepackage{title/titlestructure}

%% Utils
\usepackage{background/utils}


\title{A Deeper Dive into Relational Algebra by Way of Adjunctions}
\author{Matteo Bongiovanni}
\date{January 2023}
\supervisor{Dr.\ Nicolas Wu}
\secondmarker{Dr.\ Steffen van Bakel}
\documenttype{MEng Individual Project}
\university{Imperial College London}
\department{Department of Computing}
\degree{Joint Mathematics and Computing}

\begin{document}
\begin{comment}
General Information
The project report (occasionally referred to as a thesis) is an extremely
important aspect of the project. It serves to show what you have achieved and
should demonstrate that:

You understand the wider context of computing by relating your choice of project,
and the approach you take, to existing products or research.
You can apply the theoretical and practical techniques taught in the course to the
problem you are addressing and that you understand their relevance to the wider
world of computing.
You are capable of objectively criticising your own work and making constructive
suggestions for improvements or further work based on your experiences so far.
As a computing professional, you operate ethically and can explain your thinking
and working processes clearly and concisely to third parties who may not be
experts in the field in which you are working.
Most of the project assessors will not have followed the project throughout and
will only have a short time to listen to a presentation or see a demonstration.
For this reason they will rely heavily on the report to judge the project

Many students underestimate the importance of the report and make the mistake of
thinking that top marks can be achieved simply for producing a good product. This
is fundamentally not the case and many projects have been graded well below their
potential because of an indifferent or poor write-up. In order to get the balance
right you should consider that the aim of the project is to produce a good report
and that software, hardware, theory etc. that you developed during the project are
merely a means to this end. Don't make the mistake of leaving the write-up to the
last minute. Ideally you should produce the bulk of the report as you go along and
use the last week or two to bring it together into a coherent document.

It is helpful to get feedback from your supervisor about your project report, but
supervisors cannot be expected to look at documents written at the last minute or
at more than one chapter at a time. Allow plenty of time for this

Project Report Length
The nominal maximum report length is 60 pages, plus appendices. Shorter reports
may be appropriate for many projects and longer reports can be submitted with
prior approval (see below).

The page count includes:

title page
abstract
dedications
table of contents
The page count does not include:

bibliography
appendices (but the report should make sense without the appendices)
When writing up your work be reminded of the fact that a 60-page report isn't
necessarily better than a 30-page report, in the same way that a 10,000 line
implementation isn't necessarily better 5,000 line one. Conciseness, clarity and
elegance are invaluable qualities in report writing, just as they are in
programming. Indeed, you will lose credit if it is evident to the assessors that
the report has been padded with superfluous content in order to make it appear
more substantial than the work justifies. The assessors greatly value quality over
quantity.

It is important to understand that the report is assessed on the quality of the
technical writing. Here this means your ability to document your ideas, design,
implementation, evaluation, conclusions etc. succinctly, using appropriate
scientific language, as might a researcher when writing up their findings in an
technical magazine, academic journal or conference proceedings.

If you have good reason to use more than 60 pages you should consult your
supervisor and second marker and obtain their approval in advance - they will be
required to read and assess the whole report, whereas other assessment team
members may only scan key sections.

Project Layout
The physical layout and formatting of the report is also important. A tidy, well
laid out and consistently formatted document makes for easier reading and is
suggestive of a careful and professional attitude towards its preparation. Both
LaTeX and MS Word will allow you to produce a cleanly formatted document. Many
supervisors will advise you to use LaTeX as this solves most of the formatting
problems for you. If your report is to contain a substantial number of
mathematical formulae you are strongly advised to use LaTeX.

Project Report Structure
The exact structure of your project (the chapter titles etc) is up to you. Discuss
with your supervisor what structure works best for your project

A project report will usually contain a number of chapters (Chapter 1, Chapter 2
etc). Each chapter contains Sections (2.1, 2.2 etc) and may also contain
subsections (2.1.1, 2.1.2) etc. Try to avoid too many levels of subheading - three
is usually sufficient.

Projects will usually contain all of the following elements, although these may
not exactly map to chapters.
\end{comment}

% Front matter
\begin{comment}
This should include the project title and the name of the author of the report. You can also list the name of your supervisor if you wish. 
\end{comment}

\begin{titlepage}

\newcommand{\HRule}{\rule{\linewidth}{0.5mm}} % Defines a new command for the horizontal lines, change thickness here

%----------------------------------------------------------------------------------------
%	LOGO SECTION
%----------------------------------------------------------------------------------------

\includegraphics[width=8cm, draft=false]{title/logo.eps}\\[1cm] % Include a department/university logo - this will require the graphicx package
 
%----------------------------------------------------------------------------------------

\center\ % Center everything on the page

%----------------------------------------------------------------------------------------
%	HEADING SECTIONS
%----------------------------------------------------------------------------------------

\makeatletter
\textsc{\LARGE \@documenttype}\\[1.5cm]
\textsc{\Large \@university}\\[0.5cm]
\ifdefined\@degree\
\textsc{\large \@degree}\\[0.5cm]
\fi
\textsc{\large \@department}\\[0.5cm]
\makeatother

%----------------------------------------------------------------------------------------
%	TITLE SECTION
%----------------------------------------------------------------------------------------
\makeatletter
\HRule\\[0.4cm]
{ \huge \bfseries \@title}\\[0.4cm] % Title of your document
\HRule\\[1.5cm]
 
%----------------------------------------------------------------------------------------
%	AUTHOR SECTION
%----------------------------------------------------------------------------------------

\begin{minipage}{0.4\textwidth}
\begin{flushleft} \large
\emph{Author:}\\
\@author\ % Your name
\end{flushleft}
\end{minipage}
~\begin{minipage}{0.4\textwidth}
\begin{flushright} \large
\emph{Supervisor:} \\
\@supervisor\\[1.2em]

\ifdefined\@secondmarker\
\emph{Second Marker:} \\
\@secondmarker\
\fi

\end{flushright}
\end{minipage}\\[2cm]

%----------------------------------------------------------------------------------------
%	DATE SECTION
%----------------------------------------------------------------------------------------

{\large \@date}\\[2cm] % Date, change the \today to a set date if you want to be precise

\vfill % Fill the rest of the page with whitespace
\makeatother

\end{titlepage}
\begin{comment}
The abstract is a very brief summary of the report's contents. It should be no more than half a page long. Somebody unfamiliar with your project should have a good idea of what it's about having read the abstract alone and will know whether it will be of interest to them. Note that the abstract is a summary of the entire project including its conclusions. A common mistake is to provide only introductory elements in the abstract without saying what has been achieved.
\end{comment}


\begin{abstract}
In order to evaluate alternative equijoin implementation in Haskell, this paper
introduced the \relation{JOINBENCH} relation and surrounding methodology and
tooling. The \relation{JOINBENCH} relation is a scheme designed for the
synthesis of data optimised to help evaluate the performance of equijoins in a
variety of scenarios. Furthermore, this paper presents a low-level library that
helps users customise and define their own synthetic data sources for future
benchmarking purposes.

In their distinguished paper
\relalg{}~\cite{RelationalAlgebraByWayOfAdjunctions} it was noted that the
monadic structure of bulk types can help explain most of relational algebra.
Using this structure, the authors designed a new method to facilitate
the use of monad comprehensions in an efficient implementation of equijoins of
relational databases. This project presents an implementation of such a system
and an evaluation of the performance such query optimisations carry using the
bespoke tooling described above. 
\end{abstract}

\begin{comment}
Table of contents:
This should list the main chapters, sections and subsections of your report.
Choose self-explanatory chapter and section titles and use double spacing for
clarity. If possible you should include page numbers indicating where each chapter
section begins. 
\end{comment}
\tableofcontents

% Main matter 
\chapter{Ethical Issues}
\begin{comment}
You are required to include a short discussion of ethical, legal, societal and professional issues that are relevant to your project (usually 1 to 2 pages long). This should fit in the most appropriate section of your project report, often the Background or Conclusions section.
\end{comment}
\chapter{Ethical Issues}
\begin{comment}
You are required to include a short discussion of ethical, legal, societal and professional issues that are relevant to your project (usually 1 to 2 pages long). This should fit in the most appropriate section of your project report, often the Background or Conclusions section.
\end{comment}
\begin{comment}
The central part of the report usually consists of three or four chapters detailing the technical work undertaken during the project. The structure of these chapters is highly project dependent. They can reflect the chronological development of the project, e.g. requirements, design, implementation, experimentation, optimisation, evaluation etc. although this is not always the best approach. However you choose to structure this part of the report, you should make it clear how you arrived at your chosen approach in preference to the other alternatives documented in the background. If you have built a new piece of software you should describe and justify your design and details any interesting problems with, or features of, your implementation. Integration and testing are also important to discuss in some cases. You need to discuss the content of these sections thoroughly with your supervisor.
\end{comment}
\section{Benchmark methodology and results}
\subsection{\relation{JOINBENCH} relation and queries}
The \relation{JOINBENCH} relation is a fundamental step in the design of
the benchmark as it singlehandedly determines the possible complexity of the
possible queries. In this regard I think the \relation{JOINBENCH} was mostly
sufficient, it was largely based on previously designed benchmarks with slight
modifications to make it more specialised for the type of interesting
selection and join based queries that would be interesting in this project. One
attribute and family of queries that I found to be missing during my analysis,
however, revolved around the idea of letting the local Cartesian products grow
to be the same size as a normal product on the relations. This could be solved
by adding an attribute \relationAttribute{oneHundredPercent} whose domain only
contained one value (thus containg 100\% of the relation). I believe that this
would more interestingly highlight the disadvantages of the indexed approach as
it is likely to be reduced to a normal product. Admittedly I speculate that the lack of a need
for a filter on $n^2$ elements would still make the indexed equijoin dominant
over the product equijoin; however, I believe it would be much closer to the
comprehension equijoin. Alternatively, this could be done by conducting a
selection on an existing attribute and then joining that attribute to itself,
but I believe this goes against the design decisions expressed in
\fref{sec:background:benchmarkbestpractices}. Taken to the extreme, all
attributes in the relation could be created by various selections on the
\relationAttribute{onePercent} but this not only makes standardising relation
cardinality more difficult, it destroys the readability synthetic data sets are
meant to express. It may be interesting to include attributes in future that
cannot be derived from \relationAttribute{onePercent}, for instance other
multiples such as \relationAttribute{oneThird}, but I am not sure how much more
information this would provide in this case. These suggestion may be taken
forward as future work on designing the \relation{JOINBENCH} relation.

\subsection{Results}


\chapter{Evaluation}
\begin{comment}
Be warned that many projects fall down through poor evaluation. Simply building a system and documenting its design and functionality is not enough to gain top marks. It is extremely important that you evaluate what you have done both in absolute terms and in comparison with the state of the art. This might involve quantitative evaluation, for example based on numerical results, performance etc. or something more qualitative such as expressibility, functionality, ease-of-use etc., possible with respect to a target user base. The evaluation, often coupled with a discussion of future work (see below), should make clear the strengths and weaknesses of what you have done. It is important to understand that there is no such thing as a perfect project. Even the very best pieces of work have their limitations, so you are expected to provide a proper critical appraisal.
\end{comment}
\chapter{Ethical Issues}
\begin{comment}
You are required to include a short discussion of ethical, legal, societal and professional issues that are relevant to your project (usually 1 to 2 pages long). This should fit in the most appropriate section of your project report, often the Background or Conclusions section.
\end{comment}
\chapter{Conclusion}
\begin{comment}
Conclusions and Future Work
The project's conclusions should summarise the key insights that have been gained, be they positive or negative. For example, "The use of overloading in C++ provides a very elegant mechanism for transparent parallelisation of sequential programs", or "The overheads of linear-time n-body algorithms makes them computationally less efficient than O(n log n) algorithms for systems with less than 100000 particles". Avoid tedious personal reflections like "I learned a lot about C++ programming...". It is common to finish the report by listing ways in which the project can be taken further. This might, for example, be a plan for doing the project better if you could do a re-run, turning the project deliverables into a more polished end product, or extending the project into a programme for an MPhil or PhD.
\end{comment}

% Back matter
\bibliographystyle{unsrtnat}
% DoC uses the Vancouver Referencing Format.
\bibliography{bibs/combined}

\end{document}
